\documentclass{article}

\usepackage{amssymb}
\setcounter{tocdepth}{3}
\usepackage{graphicx}
\usepackage{graphicx}
\usepackage{subfigure}
\usepackage{multirow}
\usepackage{tabularx}
\usepackage{url}
\usepackage{rotating}
\usepackage{float}


\usepackage{amsmath}
\usepackage{graphicx}
\usepackage{subfigure}
\usepackage{multirow}
\usepackage{amsfonts}
\usepackage{graphicx}
\usepackage{subfigure}
\usepackage{multirow}
\usepackage{multicol}
\usepackage[parfill]{parskip}
\usepackage{graphicx}
\graphicspath{ {./images/} }

\usepackage[letterpaper,top=1in,bottom=1in,right=1in,left=1in]{geometry}

\usepackage{fancyhdr}
\usepackage{hyperref}
\pagestyle{fancy}
\lhead{Kistler, Johnston, Wahlbrink}
\chead{EE 480}
\rhead{Assignment 3: Pipelined PinKY}
\lfoot{}
\cfoot{}
\rfoot{Page \thepage}


\begin{document}
\title{Assignment 3: Pipelined PinKY}

\author{Ben Kistler, Jordan Johnston, Reid Wahlbrink}


\maketitle
\thispagestyle{empty}

\begin{multicols}{2}
\section{Abstract}
The assignment is to build a pipelined implementation of PinKY. The purpose of this assignment is to expand on what we did in
the last assignment and make a more complex processor that takes advantage of a pipelined implementation and learn more
about problems like control flow, and managing variables and registers in a different way than before. 


\section{General Approach}
Our processor is made up of four stages, each stage is implemented in separate modules that are instantiated in a processor
module. The pipeline is laid out in a similar way to the one described in the project description and what we went over in class. See Figure 1 for a visual overview of the design we attempted to implement.
Stage 0 owns the program counter as well as the instruction memory. It increments the PC and feeds the instruction register
into the next stage. Stage 1 owns the PRE-instruction and at first it owned the regfile as well. We ultimately moved the regfile outside of stage 1 to the processor module. Stage 1 splits up the instruction from the
instruction register and then determines what happens going forward. It then determines if PRE is needed and applies it. After
that it feeds the PC into the next stage, as well as the instruction. Stage 2 owns the ALU and carries out each instruction as it is
fed into this stage. It reads the data from memory, executes the instruction then outputs the result and the address of the result.
Stage 3 owns Z and connects everything together. It pulls the destination register, condition code and the opcode from the
instruction, with that information it then checks if the instruction is an S or not and sets Z accordingly. Finally, it writes the result
of the ALU from stage 2 into the Regfile and moves back to stage 0 to continue the pipeline. Halt is fed through each stage and
then outputted from stage 3 to the processor module so that when it is called the pipeline completes the instructions still on
going. The PC is also fed through each stage so that it increments correctly. Throughout the design we take advantage of no
ops to ensure accuracy. 
  
  \section{Issues}
The first issue we ran into was the PC was incrementing at the same time as the instruction word (ir) because we were using $<=$
on both of them in an always @(posedge clk) begin/end block. This meant we were "missing" the 0th instruction in instruction
memory. We changed ir to a wire and switched it to continual assign instead of procedural and it resolved the issue. For stage 1
we have a macro that handles the PRE instruction, it checks to see if there is an immediate in the instruction then checks if
there is a PRE. If there is both it concatenates the immediate and the PRE constant. Else it sign extends the immediate. We
had a problem at first where we weren't getting the correct value when concatenating the PRE and IMM values, but later
realized it was a syntax error and rearranged the variable being concatenated. Another problem we had implementing the PRE
was the PRE value was being recorded with every instruction, to fix this we set an if statement to check when it should be
recorded and when not. The PRE in general gave us a bunch of problems, especially with the macro. The main problems just
came from handling the syntax. To store the data to the register we had to do the data store first then set value\_out, the other
way around it gave an error. And we were getting bad results on the result and found out that it was the order of the MACRO
concatenation, once we swapped the parameters it solved the problem. We inserted No-ops to fix the issues that came from
having subsequent instructions with the same destination registers, to give the system time to properly update the Regfile and
yield the correct results. Other than that we had to resolve issues that involved the instruction set. For example, we thought our
logic to handle NE and EQ was wrong, but it turned out our AIK spec and $`$DEFINES in Verilog just didn't match. One of them
had NE EQ and the other had EQ NE so the logic was fine, we just had to make our AIK spec match our Verilog code. We ended up handling data
dependences by also changing our AIK spec to insert 3 no-ops after every instruction (w/ the exception of SYS) resulting is us
correctly handling Z, NE and EQ. We were not able to properly implement the control flow, we tried putting an array that would
track which registers were in the pipeline. In stage zero we would check Rd and insert 4 into the array because it would take
4 clock cycles for that Rd to move through the pipeline. With each clock cycle we would then decrement that value. The thought
was to check what registers were in the pipeline, and insert no-ops when needed. The array wasn't an input or output so we
couldn't view it using GTK wave and couldn't see if it was working properly. We stopped there due to time constraints. The
second idea was to in addition of feeding the instruction through each stage, we considered tracking each instruction word in
stage zero. The thought was to check Rd using the instruction word stage 3, and if it was 15 then we need to jump the PC. To
do this we would send the PC from stage 3 back to stage 1 and use that instead of the normally incremented PC
from stage 1. 

  
  \section{Testing}
To test this project, we used the testbench from the previous assignment and used GTK wave to look at the behavior of our
processor. We compiled our code using iVerilog and simulated it using VVP. We used the instruction set made by Reid and his group. To get results that we could use we inputted an empty
Regfile and then a file containing commands to run through the processor. We tested several commands to ensure that we
were getting the correct results and it was behaving the way we wanted. We payed extra attention to PRE, instructions with S
and subsequent instruction storing the result to the same reg. The purpose of this was to ensure that functions like PRE, Z, and
conditional instructions all worked properly. 
  
\end{multicols}

\begin{figure}[h]
\label{fig:1}
\caption{Attempted Processor Design}
\centering
\includegraphics[width=\textwidth]{diagram.png}
\end{figure}


\end{document}